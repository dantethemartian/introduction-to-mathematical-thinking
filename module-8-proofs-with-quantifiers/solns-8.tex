\documentclass[11pt]{exam}
\usepackage{amsfonts}
\usepackage{amsmath}
\usepackage{amssymb}
\usepackage{amsthm}
\usepackage{enumerate}
\usepackage{enumitem}
\usepackage{float}
\usepackage{geometry}
\usepackage{graphicx}
\usepackage{subcaption}
\usepackage{tikz}

\author{@dante}

\headrule{}

\begin{document}

\lhead{\textbf{Intro to Math Thinking Fall 2024: Assignment 8}}

\begin{enumerate}[leftmargin=0pt]

\item[1.] Let $m, n \in \mathbb{Z}$ such that $m = 0$ and $n = 1$. Then $m^2 + mn + n^2 = 1 = 1^2$, which is a perfect square. Hence, the statement is true.

\item[2.] Suppose $(\forall m \in \mathbb{N})(\exists n \in \mathbb{N})[PerfectSquare(mn + 1)]$. Let $n = m + 2$ such that $mn + 1 \Leftrightarrow m(m + 2) + 1 \Leftrightarrow m^2 + 2m + 1 \Leftrightarrow (m + 1)^2$. Since $m \in \mathbb{N}$, $n = m + 2 \in \mathbb{N}$. Hence, the statement is true.

\item[3.] Suppose $(\forall n \in \mathbb{N})(\exists b, c \in \mathbb{N})[Composite(f(n) = n^2 + bn + c)]$. Let $n \in \mathbb{N}$ and $f(n) = (n + 1)(n + 2) = n^2 + 3n + 2$ where $b = 3$ and $c = 2$. Hence, the statement is true.

\item[4.] Suppose $(\forall n = 2k, k \in \mathbb{N} \backslash \{1\})[(\exists p_1, p_2 \in \mathbb{P})(n = p_1 + p_2)] \Rightarrow (\forall m = 2k + 1, k \in \mathbb{N} \backslash \{1, 2\})(\exists p_1, p_2, p_3 \in \mathbb{P})[m = p_1 + p_2 + p_3]$. If $n > 5$, then $n = 2k + 3$ where $k > 1$. Since $2k > 2$ by Goldbach Conjecture, $2k = p + q$ where $p, q$ are primes. Then $n = p + q + 3$, which is the sum of 3 primes. Hence, the statement is true.

\item[5.] \textbf{Prove} $S(n) = \sum_{i = 1}^{n} 2i - 1 = n^2$ \\
Proof. Prove by induction. The base case is $S(1) = 2(1) - 1 = 1 = 1^2$, which is true. Assume for some $k$, $S(k)$ is true, and prove $P(k + 1)$. For $S(k) = 1 + 3 +...+ (2k - 1)$, the next term is obtained by adding $(2k + 1)$. By the induction hypothesis, $S(k) = k^2$, and by adding the next term, we get $k^2 + 2k + 1 = (k + 1)^2$, which is $S(k + 1)$. Hence, by induction, we have proved $S(n) \forall n \in \mathbb{N}$. $\blacksquare$

\item[6.] \textbf{Prove} $(\forall n \in \mathbb{N}) : P(n) = \sum_{r = 1}^n r^2 = \frac{1}{6}n(n + 1)(2n + 1)$ \\
Proof. Prove by induction. The base case is $P(1) = 1^2 = \frac{1}{6}(1)(1 + 1)(2(1) + 1) \Leftrightarrow 1^2 = 1 \Leftrightarrow 1 = 1$, which is true. Assume for some $k$, $P(k)$ is true, and prove $P(k + 1)$. For $P(k) = 1^2 + 2^2 + 3^2 +...+ k^2$, the next term is obtained by adding $(k + 1)^2$. By the induction hypothesis, $S(k) = \frac{1}{6}k(k + 1)(2k + 1)$, and by adding the next term, we get
\begin{align*}
    \frac{1}{6}k(k + 1)(2k + 1) + (k + 1)^2 &= \frac{1}{6}(k + 1)[k(2k + 1) + 6(k + 1)] \text{ (factor out $\frac{1}{6}(k + 1)$)} \\
    &= \frac{1}{6}(k + 1)[2k^2 + 7k + 6] \text{ (simplify)} \\
    &= \frac{1}{6}(k + 1)(k + 2)(2k + 3) \\
    &= \frac{1}{6}(k + 1)((k + 1) + 1)(2(k + 1) + 1) \text{ (rearrange)}
\end{align*}
which is $P(k + 1)$. Hence, by the induction, we have proved $P(n) \forall n \in \mathbb{N}$. $\blacksquare$

\section{OPTIONAL PROBLEMS}

\item[1.] Let $n, N \in \mathbb{N}$ such that $1 + 2 +...+ n = N$. Note we can reverse the order of addition to get the same answer such that $n + (n - 1) +...+ 1 = N$. Adding these 2 equations together we get $(n + 1) + (n + 1) +...+ (n + 1) = 2N$. Since there are $n$ terms in the sum, we can write
\begin{align*}
    2N &= n(n + 1) \\
    N &= \frac{1}{2}n(n + 1) \text{ (divide by 2)}
\end{align*}
Hence, $1 + 2 +...+ n = \frac{1}{2}n(n + 1)$ is proven. $\blacksquare$

\item[2.] Prove by induction. Suppose $(\forall n \in \mathbb{N} \backslash \{1, 2\}) P(n)$, where $P(n)$ is \textit{a collection of n points that are not all collinear in a plane where a triangle having 3 points as its vertices, which contains none of the other points in its interior, can be formed}. The base case is $P(3)$, which is true because 3 non-collinear points in a plane form a triangle with no interior points. For the induction hypothesis, we assume $P(k)$ is true for some $k$. We obtain the next term by adding a point on the plane such that there are 2 cases \\
\textbf{Case 1.} The added point is \textit{not} within the triangle. Hence, $P(k + 1)$ is true. \\
\textbf{Case 2.} The added point \textit{is} within the triangle. We can form another triangle by connecting any 2 vertices of the previous triangle to the new point. This new triangle contains no points in its interior. Thus, $P(k + 1)$ is true. Hence, by induction, we proved $P(n) \forall n \in \mathbb{N} \backslash \{1, 2\}$. $\blacksquare$

\item[3.] Let $n \in \mathbb{N}$ for the following

\begin{enumerate}[label=(\alph*)]
    \item Prove by induction. The base case is $P(n = 1)$ such that $3 | (4^1 - 1) \Leftrightarrow 3 | 3$, which is true. We assume for some $k$, $P(k)$ is true, $3 | (4^k - 1)$. The next is given by $4^{k + 1} - 1 = 4*4^k - 1$. We can rewrite this as $4*4^k - 4 + 3 = 4(4^k - 1) + 3$. We know the second term is divisible by 3 and the first term is divisible by 3 by the induction hypothesis. Hence, by induction, we proved $P(n) \forall n \in \mathbb{N}$.
    \item Prove by induction. The base case is $P(n = 5)$ such that $6! = 720 > 2^8 = 256$, which is true. Assume for some $k$, $P(k)$ is true such that $(k + 1)! > 2^{k + 3}$. To get the next term, multiply both sides by $(k + 2)$ to obtain $(k + 2)(k + 1)! = (k + 2)! > (k + 2)*2^{k + 3} = k*2^{k + 3} + 2*2^{k + 3} = k*2^{k + 3} + 2^{k + 4} > 2^{(k + 1) + 3}$, which is $P(k + 1)$. Hence, by induction, we proved $P(n) \forall n \in \mathbb{N} \backslash \{1, 2, 3, 4\}$. $\blacksquare$
    \item Prove by induction. The base case is $P(n = 1)$ such that $1*1! = 2! - 1 \Leftrightarrow 1 = 1$, which is true. Obtain the next term by adding $(k + 1)*(k + 1)!$ to $P(k)$. We get $(k + 1)! - 1 + (k + 1)(k + 1)! = (k + 1)![1 + (k + 1)] - 1 = (k + 2)! - 1$, which is $P(k + 1)$. Hence, by induction, we proved $P(n) \forall n \in \mathbb{N}$. $\blacksquare$
\end{enumerate}

\end{enumerate}
\end{document}