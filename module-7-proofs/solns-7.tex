\documentclass[11pt]{exam}
\usepackage{amsfonts}
\usepackage{amsmath}
\usepackage{amssymb}
\usepackage{amsthm}
\usepackage{enumerate}
\usepackage{enumitem}
\usepackage{float}
\usepackage{geometry}
\usepackage{graphicx}
\usepackage{subcaption}
\usepackage{tikz}

\author{@dante}

\headrule{}

\begin{document}

\lhead{\textbf{Intro to Math Thinking Fall 2024: Assignment 7}}

\begin{enumerate}[leftmargin=0pt]

\item[1.] False. Counterexample. A penguin is a bird that cannot fly.

\item[2.] False. Counterexample. Let $x = y = 1$. Then $(x - y)^2 = (1 - 1)^2 = 0 \not > 0$.

\item[3.] True. Let $x, y \in \mathbb{Q}$ where $x < y$. Let $x = \frac{p}{q}$ and $y = \frac{r}{s}$ where $p, q, r, s \in \mathbb{Z}$ and $q \not = 0$ $\land$ $s \not = 0$. Then
\begin{align*}
    \frac{x + y}{2} &= \frac{\frac{p}{q} + \frac{r}{s}}{2} \text{ (substitute)} \\
    &= \frac{ps + qr}{2qs} \text{ (simplify)}
\end{align*}
Since $\frac{ps + qr}{2qs} \in \mathbb{Q}$, and $x < \frac{x + y}{2} < y$, there exists a third rational between any 2 unequal rationals.

\item[4.] Constructing a truth table for $[(\phi \Rightarrow \psi) \land (\psi \Rightarrow \phi)] \Rightarrow [\phi \Leftrightarrow \psi]$ would result in a tautology.

\item[5.] $[(\neg \phi) \Rightarrow (\neg \psi)] \Leftrightarrow [\psi \Rightarrow \phi]$ is the contrapositive. Then from $\textbf{4.}$, $[(\phi \Rightarrow \psi) \land (\neg \phi \Rightarrow \neg \psi)] \Rightarrow [\phi \Leftrightarrow \psi]$ would result in a tautology.

\item[6.] Suppose $\$2M$ is split evenly among 5 investors. $\$2M / 5 = \$400000$, which is the most each investor will receive if split evenly. Thus, at least 1 investor receives at least $\$400000$.

\item[7.] Suppose $\sqrt{3}$ is rational. Then there exists $p, q \in \mathbb{N}$ such that $p, q$ have no common factors.
\begin{align*}
    \sqrt{3} &= \frac{p}{q} \\
    3 &= \frac{p^2}{q^2} \text{     (square)}\\
    3q^2 &= p^2 \text{     (rearrange)}
\end{align*}
$p^2$ is a multiple of 3, and thus $p$ is a multiple of 3. Let $r \in \mathbb{N}$ such that $p = 3r$.
\begin{align*}
    3q^2 &= (3r)^2 \text{ 
        (substitute)}\\
    3q^2 &= 9r^2 \text{  
        (simplify)}\\
    q^2 &= 3r^2 \text{
        (divide by 3)}
\end{align*}
$q^2$ is a multiple of 3, and thus $q$ is a multiple of 3. $p, q$ both are a multiple of 3, however $p, q$ have no common factors, which is a contradiction. Thus, $\sqrt{3}$ is irrational. $\blacksquare$

\item[8.]
\begin{enumerate}[label=(\alph*)]
    \item If the Yuan rises, the Dollar will fall.
    \item $(\forall x, y \in \mathbb{R}) [(-y < -x) \Rightarrow (x < y)]$
    \item If 2 triangles have the same area, they are congruent.
    \item If $b^2 \geq 4ac$, then $ax^2 + bx + c = 0$ (where $a, b, c, x \in \mathbb{R}$ and $a \neq 0$).
    \item If the opposite angles of quadrilateral $ABCD$ are pairwise equal, then the opposite sides are pairwise equal.
    \item If all 4 angles of quadrilateral $ABCD$ are equal, then all 4 sides are equal.
    \item $(\forall n \in \mathbb{N}) [3 | (n^2 + 5) \Rightarrow \neg (3 | n)]$
\end{enumerate}

\item[9.]
\begin{enumerate}[label=(\alph*), start=2]
    \item og: True. $(\forall x, y \in \mathbb{R})[x < y \Rightarrow -y < -x]$  \\
    conv: True. $(\forall x, y \in \mathbb{R})[-y < -x \Rightarrow x < y]$ \\
    \\
    $(\forall x, y \in \mathbb{R})[x < y \Leftrightarrow 0 < y - x \Leftrightarrow -y < -x ]$. Thus, equivalent.
    \item og: True. Congruent triangles are the same size and shape, thus will have the same area. \\
    conv: False. The triangles could have the same area but not be congruent. \\
    
    Not equivalent.
    \item Suppose $(\forall b, c \in \mathbb{R} \land \forall a \in \mathbb{R} \backslash \{0\})$ and $\exists x \in \mathbb{R}$ such that $ax^2 + bx + c = 0 \Leftrightarrow x^2 + \frac{b}{a}x + \frac{c}{a} = 0 \Leftrightarrow (x + \frac{b}{2a})^2 + \frac{c}{a} = \frac{b^2}{4a^2} \Leftrightarrow (x + \frac{b}{2a})^2 = \frac{b^2 - 4ac}{4a^2} \Leftrightarrow x + \frac{b}{2a} = \pm \frac{\sqrt{b^2 - 4ac}}{2a} \Leftrightarrow x = - \frac{b \pm \sqrt{b^2 - 4ac}}{2a}$. Since $b^2 - 4ac \geq 0 \Leftrightarrow b^2 \geq 4ac$. Thus, a solution exists if $b^2 \geq 4ac$, and the converse is also true. Thus, equivalent.
    \item If opposite sides of quadrilateral $ABCD$ are pairwise equal, it is a parallelogram. Then the opposite angles are pairwise equal. The converse is also true. Thus, equivalent.
    \item og: False. Counterexample. $ABCD$ is a rhombus. \\
    conv: False. $ABCD$ is a rectangle. \\
    \\
    Not equivalent.
    \item If $(n \in \mathbb{N})$ and $3 \nmid n$, then $n = (3k + 1) \lor (3k + 2)$ where $k \in \mathbb{Z}$. Then $n^2 + 5 = [3(3k^2 + 2k + 2) \lor 3(3k^2 + 4k + 3)]$. Thus, $3 \mid (n^2 + 5)$. If $3 \mid (n^2 + 5)$, then $n^2 + 5 = (3k + 1) \lor (3k + 2)$. Consider the case where $n^2 + 5 = 3k + 2$. Then $n^2 = 3k - 3 = 3(k - 1)$. Therefore, $3 \mid n^2$, and by Euclid's lemma, $3 \mid n$. Thus, the converse is false, and the statement and its converse are not equivalent.
\end{enumerate}

\item[10.] $(\Rightarrow)$ If $n \in \mathbb{N}$ and $12 \mid n$, then $(\forall k \in \mathbb{Z})[n = 12k]$. Then $n^3 = 12^3 k^3$ that can be written as $12 \mid n^3$. \\
$(\Leftarrow)$ False. Counterexample. Let $n = 6$ such that $n^3 = 216$. Since $216 = 12 * 18$, it can be written $12 \mid n^3$. However, $12 \nmid n$, which is a contradiction. \\
Thus, the statement is false. 

\item[11.]
$1, 2, 5$ are irrational. \\
\\
For the following, let $r, s$ be irrationals.
\begin{enumerate}
    \item[1.] Let $r + 3$ be rational such that $p, q \in \mathbb{Z}$ where $p, q$ have no common factors. Then $r + 3 = \frac{p}{q} \Leftrightarrow r = \frac{p}{q} - 3 \Leftrightarrow r = \frac{p - 3q}{q} \in \mathbb{Q}$. But $r$ was assumed to be irrational, which is a contradiction. Thus, $r + 3$ must be irrational. 
    \item[2.] Let $5r$ be rational such that $p, q \in \mathbb{Q}$ where $p, q$ have no common factors. Then $5r = \frac{p}{q} \Leftrightarrow r = \frac{p}{5q} \in \mathbb{Q}$. But $r$ was assumed to be irrational, which is a contradiction. Thus, $5r$ is irrational.
    \item[3.] Counterexample. Let $r = -s = \sqrt{2}$. Then $r + s = \sqrt{2} - \sqrt{2} = 0 \in \mathbb{Q}$. Thus, $r + s$ could be rational.
    \item[4.] Counterexample. Let $r = s = \sqrt{2}$. Then $rs = \sqrt{2} * \sqrt{2} = 2 \in \mathbb{Q}$. Thus, $rs$ could be rational.
    \item[5.] Let $\sqrt{r}$ be rational such that $p, q \in \mathbb{Q}$ where $p, q$ have no common factors. Then $\sqrt{r} = \frac{p}{q} \Leftrightarrow r = \frac{p^2}{q^2} \in \mathbb{Q}$. But $r$ was assumed to be irrational, which is a contradiction. Thus, $\sqrt{r}$ is irrational. 
    \item[6.] Consider \\
    Case 1. If $\sqrt{2}^{\sqrt{2}}$ is rational, let $r = s = \sqrt{2}$. \\
    Case 2. If $\sqrt{2}^{\sqrt{2}}$ is irrational, let $r = \sqrt{2}^{\sqrt{2}}$ and $s = \sqrt{2}$. Then $r^s = (\sqrt{2}^{\sqrt{2}})^{\sqrt{2}} = \sqrt{2}^2 = 2 \in \mathbb{Q}$. Thus, $r^s$ is rational.
    
\end{enumerate}

\item[12.]

For the following, let $m, n \in \mathbb{Z}$.

\begin{enumerate}[label=(\alph*)]
    \item Let $m, n$ be even such that $\forall q, r \in \mathbb{Z}, m = 2q$, and $n = 2r$. Then $m + n = 2q + 2r = 2(q + r)$, or $2 \mid (m + n)$. Thus, $m + n$ is even.
    \item Let $m, n$ be even such that $\forall q, r \in \mathbb{Z}, m = 2q$, and $n = 2r$. Then $mn = (2q)(2r) = 4qr$, or $4 \mid mn$. Thus, $mn$ is divisible by 4.
    \item Let $m, n$ be odd such that $\forall q, r \in \mathbb{Z}, m = 2q + 1$, and $n = 2r + 1$. Then $m + n = (2q + 1) + (2r + 1) = 2q + 2r + 2 = 2(q + r + 1)$, or $2 \mid (m + n)$. Thus, $m + n$ is even.
    \item Let $m$ be even and $n$ be odd such that $\forall q, r \in \mathbb{Z}, m = 2q$, and $n = 2r + 1$. Then $m + n = 2q + 1 + 2r = 2(q + r) + 1$, or $\neg 2 \mid (m + n)$. Thus, $m + n$ is odd.
    \item Let $m$ be even and $n$ be odd such that $q, r \in \mathbb{Z}, m = 2q$, and $n = 2r + 1$. Then $mn = (2q)(2r + 1) = 2(2qr + q)$, or $2 \mid (m + n)$. Thus, $mn$ is even.
\end{enumerate}

\section{OPTIONAL PROBLEM}
\begin{enumerate}[label=(\alph*)]
    \item True. Let $x, y \in \mathbb{R}$ such that $x = 0$ and $(\forall y \in \mathbb{R}) \mid [x + y = y]$.
    \item True. Let $x, y \in \mathbb{R}$ and $\forall x \exists y (x + y = 0)$ where $x = - y$.
    \item False. Let $a, b, c \in \mathbb{Z}$ such that $a \mid bc$. Let $a = 4$ and $b = c = 2$. Then $a \mid bc \Leftrightarrow 4 \mid (2*2) \Leftrightarrow 4 \mid 4$, but $(a \nmid b \lor a \nmid c) \Leftrightarrow (4 \nmid 2 \lor 4 \nmid 2)$.
    \item True. Let $x \in \mathbb{R}$ and $y \in \mathbb{R} \backslash \mathbb{Q}$. Let $x + y$ is rational such that $p, q \in \mathbb{Q}$ where $p, q$ have no common factors. Then $x + y = \frac{p}{q} \Leftrightarrow y = \frac{p}{q} - x \Leftrightarrow y = \frac{p - qx}{q} \in \mathbb{Q}$. But $y$ is irrational, which is a contradiction. Thus, $x + y$ is irrational.
    \item True.
    Case 1. Let $x \in \mathbb{R}$ and $y \in \mathbb{R} \backslash \mathbb{Q}$. See $\textbf{(d)}$.
    \\
    Case 2. Let $x, y \in \mathbb{R}$ such that $p, q \in \mathbb{Z}$ where $p, q$ have no common factors. Then $x + y = \frac{p}{q} \Leftrightarrow x = \frac{p}{q} - y \Leftrightarrow x = \frac{p - qy}{q} \in \mathbb{Q}$. Thus, $x + y \in \mathbb{Q} \in \mathbb{R}$.
    \\
    Case 3. See $\textbf{(11.3)}$. \\
    Thus, at least 1 of $x, y$ is irrational.
    \item False. Counterexample. Let $x, y \in \mathbb{R} \backslash \mathbb{Q}$ such that $x = -y = \sqrt{2}$. Then $x + y = \sqrt{2} - \sqrt{2} = 0 \in \mathbb{Q}$. Thus, $x + y$ could be rational without either $x, y \in \mathbb{Q}$.
        
\end{enumerate}

\end{enumerate}
\end{document}